% Adam DeConinck's Resume

\documentclass[10pt]{article}
\usepackage{geometry}
\usepackage[T1]{fontenc}
\usepackage{multicol}

\pagestyle{empty}
\geometry{letterpaper,tmargin=0.5in,bmargin=0.5in,lmargin=0.5in,rmargin=0.5in,headheight=0in,headsep=0in,footskip=0in}

\setlength{\parindent}{0in}
\setlength{\parskip}{0in}
\setlength{\itemsep}{-0.2cm}
\setlength{\topsep}{0in}
\setlength{\tabcolsep}{0in}

%\hyphenation{encapsulations, Scholars, lithography, Engineering, microscopy, assistance}

% Name and contact information
\newcommand{\name}{Adam DeConinck}
\newcommand{\addr}{909 S. 5th St \#139, Champaign, IL 61820}
\newcommand{\phone}{(347) 709-2326}
\newcommand{\email}{ajdecon@ajdecon.org}
\newcommand{\website}{http://www.ajdecon.org/}
\newcommand{\linkedin}{http://www.linkedin.com/in/ajdecon}
\newcommand{\twitter}{http://www.twitter.com/ajdecon}

% This defines how the name looks
\newcommand{\bigname}[1]{
	\begin{center}\fontfamily{phv}\selectfont\Huge\scshape#1\end{center}
}



% Section heading.
\newcommand{\bigsection}[1]{	
	\vspace{4pt}
	{\fontfamily{phv}\selectfont\Large#1}

	\vspace{-10pt} \rule{\textwidth}{1pt}
}

% Section heading.
\newcommand{\smallsection}[1]{	
	\vspace{2pt}
	{\fontfamily{phv}\selectfont\Large#1}
}

\newcommand{\jobitem}[4]{
    \vspace{2pt}

    \textbf{#1 @ #2} \hfill #3 %\\ #4

    \vspace{-8pt}
}




\begin{document}

\fontfamily{ppl} \selectfont

% Name with horizontal rule
\bigname{\name}

\vspace{-8pt} \rule{\textwidth}{1pt}

\vspace{-1pt} {\small\itshape \addr \hfill \phone \hfill \email \hfill \website} % \\ \website \hfill \linkedin}

\vspace{6pt}
%%%%%%%%%%%%%%%%%%%%%%%%%%%%%%
\bigsection{Summary}
Engineer and scientist with expertise in high-performance computing and cloud computing. 
Background in academic and commercial research with a heavy focus on computational analysis. Interested in ``Big Data'' technologies and the application of cloud computing to HPC for non-traditional users. 
% Proficient
%n rapid deployment of dynamically-allocated computational resources, and tuning cluster configuration for specific applications.  %Excellent \\communication skills and ability to support 
%researchers on HPC systems.

\vspace{4pt}
%%%%%%%%%%%%%%%%%%%%%%%%%%%%%%%%
\bigsection{Professional Experience}

\vspace{-10pt}

\jobitem{Systems Administrator/Applications Specialist}{R Systems NA, Inc.}{November 2010--present}{Champaign, IL}

\begin{itemize}\setlength{\itemsep}{0cm}
  \setlength{\parskip}{0cm}

%\item Engineer/sysadmin at a large, complex site providing on-demand access to private HPC clusters.
\item Design, deploy, and support private ``cloud'' HPC clusters on R Systems hardware for research customers.
\item Measure and tune behavior of customer applications for best performance and usability.
\item Support and train users with varied levels of past experience with HPC.
\item Develop software and pre-built images for automating deployment of custom compute clusters.
\item Track and evaluate new software and hardware technologies for existing and potential customers.
\item Outreach activites to relevant research communities and open-source projects.
\item Active and past projects include:
    \begin{itemize}\setlength{\itemsep}{0cm}
        \setlength{\parskip}{0cm}
        \vspace{-4pt}
        \item Long-term (1+ years) deployment of 600-1000 core RHEL 5.6 cluster for meteorological modeling application.
        \item Long-term (1+ years) deployment of 1000 core Windows 2008 HPC R2 cluster for actuarial application.
        \item Deploy short-term test clusters for a high-performance storage vendor benchmarking development hardware.
        \item Over a dozen other shorter-term or smaller deployments for various scientific and engineering applications.
    \end{itemize}
\end{itemize}

%\item Design, deployment, administration and support for private, on-demand HPC clusters.
%\item Performance tuning of applications in domains including
%meteorology, computational fluid dynamics, finite element analysis, bioinformatics, geophysics and actuarial science.
%\item Project lead on two long-term HPC deployments under 24/7 production load:
%\begin{itemize}\setlength{\itemsep}{0cm}
%\setlength{\parskip}{0cm}
%\vspace{-4pt}
%\item 1000 cores, Windows HPC 2008 R2, actuarial application (MG-ALFA).
%\item 600 cores (burst capacity up to 1000), RHEL 5.6, meteorology application (WRF).
%\end{itemize}
%\vspace{-4pt}
%\item Project lead or backup on over a dozen smaller and/or short-term deployments (mix of Linux and Windows).
%\item R\&D work on automated deployment of self-service ``cloud'' HPC clusters tuned for specific applications.
%\item Outreach and development for open-source projects relevant to our technical interests.
%%\item Provision, administer and tune performance on HPC clusters running both Linux and Windows.
%%%\item Environment includes about 10,000 cores (1200 nodes) total.
%\item Experience with both short-term projects (a few days) and renewable year-long contracts.
%\item Optimize and troubleshoot applications in domains including weather forecasting, 
%computational fluid dynamics, finite element analysis, bioinformatics and actuarial science.
%\item Support and training for academic and commercial researchers with diverse requirements and experiences levels.

%\end{itemize}


\vspace{-6pt}
\jobitem{Graduate Research Assistant}{University of Illinois}{August 2007--November 2010}{Urbana, IL}

\begin{itemize}\setlength{\itemsep}{0cm}
  \setlength{\parskip}{0cm}

\item Researched problems in microfluidics, colloidal physics, 3D particle tracking and DNA genotyping.

\item Performed computational analysis of microscopy images to extract physical and chemical data using custom software.

% \item Designed and fabricated microfluidic devices for flow lithography and chemical gradients.

% \item Research group included undergraduate and graduate students, and Ph.D. scientists.

% \item Core skills used: microfabrication, microscopy, data analysis and soft matter physics.

% \item Extensive experience with image analysis using Matlab and Java.

%\item Developed software for computational data analysis and automation of research instruments.

%\item System administration for group compute servers and did general computer support

\item Trained and supervised undergraduate researchers and administered group compute servers.

%\item Maintained several actively-used research instruments, as well as research group wiki and servers.

%\item Presented results and analysis to research group every 2-3 weeks in group meetings.
%\item Participate in weekly group meetings, giving research presentations every 2-3 weeks.

%\item Assisted colleagues in areas including data analysis, programming, microfabrication and microscopy.

%\item Trained and supervised several undergraduate research assistants.

%\begin{itemize}\setlength{\itemsep}{0cm}
%  \setlength{\parskip}{0cm}
%\item 2008-2009: Graduate mentor through the Intel Scholars for Undergraduate Research program for improving representation of women in technology.
%\end{itemize}

%\item Project: controlled self-assembly of anisotropic colloidal particles
%\begin{itemize}\setlength{\itemsep}{0cm}
%  \setlength{\parskip}{0cm}
%\item Fabricated microfluidic devices using soft lithography and clean-room techniques.
%\item Assembled a microfluidic flow lithography (FL) system and built a LabView controller.
%\item Used FL to produce self-assembling particles with controlled shape and chemistry.
%\item Imaged free and self-assembled particles using 3D confocal microscopy.
%\item Developed Matlab code for morphological image analysis to study particle behavior.
%\end{itemize}
%
%\item Project: DNA genotyping with polymer micro-beads
%\begin{itemize}\setlength{\itemsep}{0cm}
%  \setlength{\parskip}{0cm}
%\item Collaboration with Chemistry Ph.D. student in Nuzzo research group.
%\item Designed and fabricated labeled DNA/polymer micro-beads by FL.
%\item Developed analysis for genotyping of sample DNA based on fluorescence data.
%\item Implemented image processing routines and genotyping analysis in Matlab.
%\item Co-wrote a publication based on these results, submitted to \textit{Angewandte Chemie}.
%\end{itemize}

\item Awarded the National Defense Science and Engineering Graduate Fellowship (2008--2010).

\end{itemize}


%\jobitem{Senior research project}{Michigan Technological University}{August 2006--May 2007}{Houghton, MI}
%
%\begin{itemize}\setlength{\itemsep}{0cm}
%  \setlength{\parskip}{0cm}
%\item Fabricated boron nitride thin films and nanotubes by pulsed-laser deposition.
%\item Characterized samples by Fourier-transform infrared spectroscopy and electron microscopy.
%\item Presented results of my research at a Physics department colloquium.
%\end{itemize}

%\jobitem{Physics Learning Center Coach}{Michigan Technological University}{August 2006--May 2007}{Houghton, MI}
%\vspace{16pt}
%\begin{itemize}\setlength{\itemsep}{0cm}
%  \setlength{\parskip}{0cm}
%\item Taught physics in a free tutoring center supplied by the Physics Department.
%\item Conducted interviews and assisted in hiring decisions for the next year's tutoring staff.
%\item Developed learning plans for one-on-one students.
%\end{itemize}

\vspace{-6pt}
\jobitem{Contractor}{Dow Corning Corporation}{May 2006--August 2006}{Midland, MI}

\begin{itemize}\setlength{\itemsep}{0cm}
  \setlength{\parskip}{0cm}
% \item Project: develop technique to evaluate thin-film encapsulations for organic LEDs.
%\item Worked on a team including other contractors and Dow Corning scientists and engineers.
\item Assembled hardware and developed software for an automated test station for testbed OLED devices. 
%\item Developed automated image analysis software for measurement of surface defects.
%\item Developed software to automate test station operation (LabView) and analysis of experimental data (Java and Visual Basic).
%\item Developed an automated report generator in VB/Excel to present raw test station data.
%\item Helped to develop image analysis algorithms to track the development of device defects. 
%\item Implemented these defect analysis algorithms in Java as an ImageJ plugin.
%\item Developed a Visual Basic macro to translate test station data into an Excel summary report.
%\item Produced detailed project documentation to facilitate the future use of these instruments.
%\item Presented my work at the end of the summer as part of an in-house seminar series.
\item OLED test station and software were still in active use as of project completion in 2009.
\end{itemize}
%

%\jobitem{Teaching Assistant}{Michigan Technological University}{August 2005--December 2005}{Houghton, MI}

%\vspace{16pt}
%
%\begin{itemize}\setlength{\itemsep}{0cm}
 % \setlength{\parskip}{0cm}
%\item Taught two sections of an introductory mechanics lab, including lab setup and grading reports.
%\item Prepared equipment and reading material for each week's lab.
%\item Graded lab reports and worked directly with students in need of help.
%\end{itemize}

\vspace{-6pt}
\jobitem{Undergraduate Research Assistant}{Michigan State University}{June 2005--August 2005}{East Lansing, MI}
\vspace{10pt}

%\begin{itemize}\setlength{\itemsep}{0cm}
%  \setlength{\parskip}{0cm}
%\item NSF-funded Research Experience for Undergraduates program.
%\item Developed software for analysis of neutron scattering data in Python.
%\item Analyzed atomic structure of magnetoresistive behaviors in materials relevant to data storage.
%\begin{itemize}\setlength{\itemsep}{0cm}
%  \setlength{\parskip}{0cm}
%	\item Colossal magnetoresistive (CMR) materials are useful in data storage applications.
%	\item Analysis revealed a molecular distortion which might help explain CMR behavior.
%\end{itemize}
%\item Automated the analysis of large datasets across multiple servers.  Assisted in network configuration.
%\item Produced a final report for the NSF on my work and helped prepare a paper published in \textit{Physical Review Letters}.
%\end{itemize}

%\vspace{16pt}
\jobitem{Undergraduate Research Assistant}{Michigan Technological University}{June 2004--May 2005}{Houghton, MI}

%%
%\begin{itemize}\setlength{\itemsep}{0cm}
%  \setlength{\parskip}{0cm}
%\item Developed fabrication and processing techniques for producing thin films of vertically-aligned carbon nanotubes
%for research on field emission devices.
%\end{itemize}
%%
%%
%%
%%\jobitem{Computer Lab Tech}{Tecumseh Public Library}{July 2001--May 2002}{Tecumseh, MI}

%\begin{itemize}\setlength{\itemsep}{0cm}
%  \setlength{\parskip}{0cm}
%\item Helped manage a public-use computer lab with 8-10 workstations.
%\item Managed upgrades and repaired/restored damaged workstations.
%\item Provided computer assitance to library patrons, from office tasks to simple web design assistance.
%\end{itemize}

%\pagebreak

\vspace{16pt}


\bigsection{Other Experience}
\vspace{-15pt}
\begin{itemize}\setlength{\itemsep}{0cm}
  \setlength{\parskip}{0cm}
    \item \textbf{Warewulf 3.0}: Developer and tester for the open-source HPC cluster manager from Lawrence Berkeley National Lab.
    \item \textbf{Presentations and Publications}: Academic publications and presentations to the research and engineering \\ 
    communities on topics in science and HPC. Full list at http://www.ajdecon.org/projects/pubs.
    \item \textbf{ImageJ plugins}: Developed mathematical morphology processing routines for the popular image-processing tool, available at http://www.github.com/ajdecon/imagej\_morphology.
    \item \textbf{Fencing}: Active \'{e}p\'{e}e fencer. Former President of the Michigan Tech Fencing Club and assistant instructor at The Point Fencing Club and School in Champaign, IL.
%    \item Presentations to the research community and publications in \textit{Physical Review Letters}, \textit{Chemistry}, and others: 
%    \\ Full list at http://www.ajdecon.org/projects/pubs.
%    \item Developed mathematical morphology plugins for the popular ImageJ image-processing tool: \\
%    Code and binaries at http://www.github.com/ajdecon/imagej\_morphology.
%	\item Strong professional interest in ``big data'', computational analysis and scientific computing.  
%    \item Attended the ``Big Data for Science'' workshop at Indiana University in July 2010.
%	\item President of the Michigan Tech Fencing Club and assistant instructor at the Point Fencing Club in Champaign, IL.
	%\item Active interest in new and creative board and card games, especially games with a strong social component.

\end{itemize}


%%%%%%%%%%%%%%%%%%%%%%%%%%%%%%%%
\bigsection{Skills and Knowledge}

\vspace{-15pt}

\begin{multicols}{2}

\begin{itemize}\setlength{\itemsep}{0cm}
\setlength{\parskip}{0cm}

\item \textbf{High Performance Computing:}
	\begin{itemize}\setlength{\itemsep}{0cm}
	\setlength{\parskip}{0cm}
	\item Experienced sysadmin in Linux and Windows
	\item Configuration management with Puppet
	\item Provisioning: Perceus/Warewulf, xCAT, Cobbler
	\item Schedulers: Torque/PBS, Grid Engine, Hadoop
	\item Virtualization: KVM, Hyper-V, EC2, OpenQRM
	\item Parallel file systems: Lustre, PVFS2, HDFS
	\item Networking: Ethernet and Infiniband
	\item Experienced with Microsoft HPC Pack
	\item Excellent user support, communications skills 
	\end{itemize}

\item \textbf{Programming Languages:}
	\begin{itemize}\setlength{\itemsep}{0cm}
	\setlength{\parskip}{0cm}
	\item Scientific programming: Matlab, Java, Python
	\item Sysadmin: Python, Perl, shell scripts
	\item Classroom or personal experience: Fortran, C, R
	\end{itemize}

\item \textbf{Materials Science:}
	\begin{itemize}\setlength{\itemsep}{0cm}
	\setlength{\parskip}{0cm}
	\item Image processing, mathematical morphology
	\item Microscopy: optical, fluorescent, SEM, confocal
	\item Microfabrication with soft materials
	\item Microfluidics and nonlinear rheology
	\end{itemize}

\end{itemize}

\end{multicols}

%
%\textit{Materials science}
%\begin{itemize}\setlength{\itemsep}{0cm}
%  \setlength{\parskip}{0cm}
%
%		\item Microfabrication using microfluidics, soft lithography, and UV photolithography.
%%chemical vapor \\deposition and pulsed-laser deposition.
%		\item Materials characterization with fluorescence and confocal microscopy, SEM, AFM, and FTIR.
%		\item Experience with polymer synthesis, photocurable materials, and rheology.
%%		\item Electronic and optical microscopy, including fluorescence and confocal microscopy.
%		\item Expertise in polymer physics, soft materials and electronic materials.
%\end{itemize}

%\textit{Computing}
%\begin{itemize}\setlength{\itemsep}{0cm}
  %\setlength{\parskip}{0cm}

		
%		\item Extensive experience with provisioning, configuration and maintenance of dynamic HPC clusters.
%        \item Linux and Windows system administration in the context of high-performance computing.
 %       \item Design, deployment and configuration of HPC clusters with Ethernet and Infiniband interconnects.
%	\item Dynamic provisioning of servers using Perceus/Warewulf, Cobbler, xCAT and Microsoft HPC Pack.
%	\item Configuration of workload management systems such as Torque/PBS, Grid Engine, Platform LSF and Hadoop.
%        \item HPC tools for Linux (Perceus/Warewulf, Cobbler, PBS, Grid Engine, Hadoop) and Windows (HPC Pack).
%        \item Configuration and administration of parallel/distributed filesystems including Lustre, PVFS2 and HDFS.
%        \item Provisioning of virtualized servers locally (Xen, KVM, Hyper-V) and in cloud systems (EC2, OpenQRM).

%        \item Network administration on ethernet and Infiniband networks
%        \item Production programming experience in Python, Perl, Java, and Matlab.
%        \item Classroom or personal programming experience in Fortran, C, R, and Scala.
%        \item Background in scientific computing, data analysis, and commercial and academic research.
%        \item Extensive experience supporting scientists, engineers, and other users with varied levels of experience.

%		\item Proficient in Python, Java, and Matlab for scientific computing, and Perl and PowerShell for scripting.
%		\item Some experience with Fortran, C, Visual Basic and R.
%        \item Extensive experience supporting scientists and engineers in commercial and academic settings.
        
%		\item Proficient in programming and data analysis with Matlab, Java and Python (numpy+ scipy+matplotlib).
%		\item Some experience with with Perl, Fortran, R, Visual Basic, MySQL and SQLite databases.
%		\item Extensive experience with computational analysis of text, numerical and image data.
		%\item Experience with setting up small computer networks (wired or wireless) and basic system administration on Linux and Windows.
%		\item Automated unit tests of analysis code in Java with JUnit and in Matlab using hand-written tests.
%		\item Programming in Java, Python, Perl, R and Visual Basic.
%		\item Setup and maintenance of PCs and servers running Microsoft Windows and Linux.
%		\item Maintenance of networks including Windows, Linux and Mac workstations.
%		\item Experience with optical, fluorescence and confocal microscopy, image processing and analysis of\\ microscopy data.
%		\item Image processing and analysis, curve fitting of fluorescence data and analysis of text data..
%		\item Experience with parallel data analysis using Hadoop or Twister on Amazon EC2 clusters.
%		\item Web development using HTML/CSS/JavaScript, PHP and Perl.


%		\item Basic understanding of computational statistics and common machine learning techniques.
%		\item Web development skills: LAMP environment, HTML/CSS, PHP, Python, Perl, Javascript.
%%Setup and maintenance of CMS systems including Wordpress and MediaWiki.
%		\item{Experienced with Perl, Fortran, Visual Basic, statistical analysis using R, parallel computing using Hadoop, and web development in LAMP environments.}





%		\item{Strong background in mathematics and scientific programming.}
%		\item{Some web design experience with HTML/CSS, Perl and Python CGI and basic JavaScript.}
%		\item{Comfortable in both Linux and Windows environments.}
		
%		\item Experienced with image processing and analysis, including image morphology.
%		\item{Strong interest in data analysis, machine learning and scientific computing.}
%		\item{Expertise in microfabrication, soft lithography, microscopy and polymer physics.}
%\end{itemize}

\vspace{-5pt}

\bigsection{Education}

\vspace{-10pt}

\textbf{M.S. in Materials Science and Engineering}--University of Illinois \hfill December 2010
%\vspace{-6pt}
%\begin{itemize}\setlength{\itemsep}{0cm}
%  \setlength{\parskip}{0cm}
%\item Focus on computational analysis of microscopy data, polymer physics and microfabrication.
%\end{itemize}

\textbf{B.S. in Physics}--Michigan Technological University \hfill May 2007
%\vspace{-6pt}
%\begin{itemize}\setlength{\itemsep}{0cm}
%  \setlength{\parskip}{0cm}
%\item Minors in Mathematics and Electronic Materials.
% \item Graduated magna cum laude.
%\end{itemize}

%Transcripts including a list of relevant classes are available on request.



%%%%%%%%%%%%%%%%%%%%%%%%%%%%%%%%%%%%
%\bigsection{Presentations and Publications}
%\vspace{-16pt}
%\begin{itemize}\setlength{\itemsep}{0cm}
%  \setlength{\parskip}{0cm}
%\item Academic publications in \textit{Physical Review Letters}, \textit{Advanced Materials} and \textit{Chemistry: A European Journal}.
%\item Presentations given to the Materials Research Society, the University of Illinois and the Society of Actuaries.
%\item Full list: http://www.ajdecon.org/projects/pubs.
%\end{itemize}
%	\item{Wu, \textbf{DeConinck}, Lewis and White,
% ``Omnidirectional printing of 3D microvascular networks using direct-write assembly'',
%\textit{Soft Matter} (in preparation).}
%	\item{Zhang, \textbf{DeConinck}, Slimmer, Lewis and Nuzzo, 
%``Genotyping by Alkaline Dehybridization Using Graphically Encoded Particles'', 
%\textit{Angewandte Chemie} (submitted, 2010).}
%
%	\item{\textbf{DeConinck}, Shepherd, Cote, Granick, Schweizer and Lewis. \textit{Programming Function via Soft Materials}, 
%``Fabrication and Self-Assembly of Janus Ellipsoids and Rods'',
%workshop at the 
%Frederick Seitz 
%Materials Research Laboratory, 
%University of Illinois, Urbana, Il (2009).}
%
%	\item{\textbf{DeConinck}, Shepherd, Cote and Lewis, 
%``Fabrication and Self-Assembly of Anisotropic Patchy Particles'',
%\textit{Materials Research Society Fall Meeting}, Boston, MA (2008).}
%	
%	\item{Bozin, Schmidt, \textbf{DeConinck}, Paglia, Mitchell, Chatterji, Radaelli, Proffen and Billinge, 
%``Understanding the Insulating Phase in Colossal Magnetoresistance Manganites: Shortening of the Jahn-Teller Long-Bond across the Phase Diagram of La$_{1-x}$Ca$_x$MnO$_3$'', 
%\textit{Physical Review Letters} \textbf{98} 137203 (2007).}
%	\item{Ulmen, Kayastha, \textbf{DeConinck}, Wang and Yap, 
%``Stability of field emission current from various types of carbon nanotube films'', 
%\textit{Diamond and Related Materials} \textbf{15} 212 (2006).}
%\end{itemize}
%
%%%%%%%%%%%%%%%%%%%%%%%%%%%%%%%%%%%
%\bigsection{Awards and Fellowships}
%
%\vspace{-16pt}
%
%\begin{itemize}\setlength{\itemsep}{0cm}
%  \setlength{\parskip}{0cm}
%	\item{National Defense Science and Engineering Graduate (NDSEG) Fellowship supported by the Army Research Office: full
%	fellowship support, 2008--present.}
%
%	\item{Graduate mentor for the Intel Scholars for Undergraduate Research program for improving representation of women in science and engineering, 2008--2009.}
%
%	\item{Michigan Technological University Scholar Award (full degree support), 2003--2007.}
%
%	\item{Michigan Technological University Provost's Award for Scholarship, 2006.}
%
%	\item{Undergraduate research fellowship from the Michigan Space Grant Consortium, 2004--2005.}
%
%	\item{National Merit Scholarship awarded by Siemens Corporation, 2003--2007.\\ \ }
%\end{itemize}
%
%\vspace{-8pt}



%%%%%%%%%%%%%%%%%%%%%%%%%%%%%%%%%%
%\bigsection{References}
%
%\vspace{-16pt}
%
%\begin{itemize}\setlength{\itemsep}{0.25cm}
%  \setlength{\parskip}{0cm}
%
%\item \textbf{Jennifer Lewis}, Director, Frederick Seitz Materials Research Laboratory, University of Illinois.\\ Graduate advisor. jalewis@illinois.edu.  (217) 333-1370.
%\item \textbf{Jacinta Conrad}, Assistant Professor, University of Houston.\\ Former post-doctoral researcher in Lewis Group. jcconrad@uh.edu. (713) 743-3829.
%\item \textbf{Elizabeth Glogowski}, Post-doctoral researcher, Lewis Group, University of Illinois.\\ emglogowski@gmail.com. (413) 687-3187.
%%\item \textbf{Vasgen Shamamian}, Associate Scientist, Dow Corning Corporation.\\ v.shamamian@dowcorning.com
%
%\end{itemize}


%%%%%%%%%%%%%%%%%%%%%%%%%%%%%%%%%%
%\bigsection{Other interests}
%\begin{itemize}\setlength{\itemsep}{0cm}
%  \setlength{\parskip}{0cm}
%
%	\item{Fencing with epee and foil.  (President of the Fencing Club at Michigan Tech, 2005--2007; current member at the Point Fencing Club in Champaign, IL.)}
%
%	\item{Jazz and classical trumpeter since middle school, in school bands and independent groups.}
%
%	\item{Strong interest in board games and card games.  Current favorites are Settlers of Catan, Dominion, Pandemic and Nuclear War.} 
%\end{itemize}
%
\end{document}
